\newcommand{\GMS}{\mathbb{G}}
\newcommand{\Gnot}{\mathsf{G^0}}
\newcommand{\tmat}{\mathsf{t}}
\newcommand{\boldr}{\boldsymbol{r}}

\begin{frame}[label=fgr]
  \frametitle{Computing X-ray Absorption from First Principles}
  \begin{columns}
    \begin{column}{0.75\linewidth}
      In XAS we measure the \alert{dipole mediated}$^\mathrm{[1]}$
      transition of an electron in a \alert{deep core}$^\mathrm{[2]}$
      state {\color{blue}$|i\rangle$} into an
      \alert{unoccupied}$^\mathrm{[3]}$ state
      {\color{red}$|f\rangle$}:

      \begin{columns}
        \begin{column}{0.15\linewidth}
          ~
        \end{column}
        \begin{column}{0.7\linewidth}
          \begin{block}{Fermi's Golden Rule}
            $\mu(E) \propto\> \sum\limits_f^{E_f>E_F}
            \big|{\color{red}\langle f|}
            {\color{blue}\hat{\epsilon}\cdot\mathbf{r}}
            {\color{blue}|i\rangle}\big|^2 \delta(E_f)$
          \end{block}
        \end{column}
        \begin{column}{0.15\linewidth}
          ~
        \end{column}
      \end{columns}

      \medskip

      Broadly speaking, there are two ways to solve this equation:
      \begin{enumerate}
      \item Accurately represent {\color{blue}$|i\rangle$}$^\mathrm{[4]}$  and
        {\color{red}$|f\rangle$}$^\mathrm{[5]}$, then evaluate
        the integral directly. This is the approach taken, for example, by
        molecular orbital theory.
      \item Use multiple scattering theory, AKA a Green's
        function$^\mathrm{[6]}$ or propagator formalism:\\
        {\footnotesize
        $\mu(E) \propto
        -\frac{1}{\pi} \Im {\color{blue}\langle \mathrm{i}
          |} \hat{\epsilon}^\ast \cdot \boldsymbol{r} \,
        {\color{Green4}\mathbb{G}(r,r';E)} \hat{\epsilon} \cdot \boldsymbol{r}' {\color{blue}|
          \mathrm{i} \rangle}
        \Theta(E-E_F)$.}
      \end{enumerate}      
    \end{column}
    \begin{column}{0.25\linewidth}
      \begin{enumerate}[1.]
        \scriptsize
      \item A photon interacts with an electron
      \item Typically a \textit{1s}, \textit{2s}, or \textit{2p} electron
      \item A bound or continuum state \textbf{not} already containing
        an electron
      \item Easy --- basic quantum mechanics
      \item Hard work, lots of computation
      \item {\color{Green4}$\GMS$} is called a Green's function.
      \end{enumerate}
    \end{column}
  \end{columns}
\end{frame}

\begin{frame}
  \frametitle{Real Space Multiple Scattering}

  In multiple scattering theory, all the hard work is in computing
  the Green's function.

  \begin{description}
  \item[${\color{Green4}\GMS}$] the function that describes all
    possible ways for a photoelectron to interact with the
    surrounding atoms
  \item[$\Gnot$] the function that describes how an electron
    propagates between two points in space
  \item[$\tmat$] the function that describes how a photo-electron
    scatters from a neighboring atom
  \end{description}

  \begin{block}{Expanding the Green's function}
    ~\\[-6ex]
    \begin{align}
      {\color{Green4}\GMS} =& \big(1 - \Gnot \tmat\big)^{-1} \, \Gnot \tag{XANES}\\
      =& \Gnot + \Gnot\,\tmat\,\Gnot +
      \Gnot\,\tmat\,\Gnot\,\tmat\,\Gnot +
      \Gnot\,\tmat\,\Gnot\,\tmat\,\Gnot\,\tmat\,\Gnot + ... \tag{EXAFS}
    \end{align}
  \end{block}
\end{frame}

\begin{frame}{Scattering Paths}
  \textbf{Full multiple scattering (XANES):} Solving {\color{DarkOrchid4}
    ${\color{Green4}\GMS} = \big(1 - \Gnot \tmat\big)^{-1} \, \Gnot$}
  considers {\LARGE ALL} paths within some cluster of atoms:
  \begin{columns}[T]
    \begin{column}{0.33\linewidth}
      \begin{center}
        {\color{DarkOrchid4}single scattering path}\\
        {\color{blue}\SS}\\
        (2 legs)
      \end{center}
    \end{column}
    \begin{column}{0.33\linewidth}
      \begin{center}
        {\color{DarkOrchid4}double scattering path}\\
        {\color{blue}\FDS}\\
        (3 legs)
      \end{center}
    \end{column}
    \begin{column}{0.33\linewidth}
      \begin{center}
        {\color{DarkOrchid4}triple scattering path}\\
        {\color{blue}\FTS}\\
        (4 legs)
      \end{center}
    \end{column}
  \end{columns}

  \medskip

  \begin{exampleblock}{EXAFS path expansion}
    The clever thing about \textsc{feff} is that each term is further
    expanded as a sum of all paths of that order.

    \medskip

    {\color{DarkOrchid4}$\Gnot\,\tmat\,\Gnot$} is expanded as a sum of
    {\color{DarkOrchid4}single scattering} paths

    \medskip

    {\color{DarkOrchid4}$\Gnot\,\tmat\,\Gnot\,\tmat\,\Gnot$} is a sum
    of all {\color{DarkOrchid4}double scattering} paths

    \medskip

    and so on.
  \end{exampleblock}

\end{frame}
